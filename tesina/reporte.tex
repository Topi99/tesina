%%%%%%%%%%%%%%%%%%%%%%%%%%%%%%%%%%%%%%%%%
% Thesis LaTeX Template Modified by Ng for the Tesina Course
% Version 1.2 (05/24/2018)
% Note:
% Make sure to include the Thesis.cls  file in the folder
%%%%%%%%%%%%%%%%%%%%%%%%%%%%%%%%%%%%%%%%%

%----------------------------------------------------------------------------------------
%	PACKAGES AND OTHER DOCUMENT CONFIGURATIONS
%----------------------------------------------------------------------------------------

\documentclass[11pt, a4paper, oneside]{Thesis} % paper size redifined in Thesis.cls
\usepackage[square, numbers, comma, sort&compress]{natbib}
\usepackage[nodayofweek]{datetime}
\usepackage{float}
\usepackage{array}
\usepackage{wrapfig}
\usepackage{pdfpages}
\usepackage[utf8]{inputenc}  % Ng Edit for accents in spanish
\usepackage[spanish]{babel}
\usepackage{amsmath}
\usepackage{algpseudocode}
\usepackage{algorithm}
\usepackage{siunitx}

\hypersetup{urlcolor=blue, colorlinks=true}
\title{\ttitle}
\thesistitle{tesina ISC}
\algnewcommand\algorithmicforeach{\textbf{for each}}
\algdef{S}[FOR]{ForEach}[1]{\algorithmicforeach\ #1\ \algorithmicdo}

\def\equationautorefname~#1\null{(#1)\null} %to use parenthesis in eqs.
\begin{document}
\frontmatter
\setstretch{1.3}

\fancyhead{}
\rhead{\thepage}
\lhead{}

\pagestyle{fancy}
\newcommand{\HRule}{\rule{\linewidth}{0.5mm}}

% PDF meta-data
\hypersetup{pdftitle={\ttitle}}
\hypersetup{pdfsubject=\subjectname}
%\hypersetup{pdfauthor=\authornames}
\hypersetup{pdfkeywords=\keywordnames}

%----------------------------------------------------------------------------------------
%	TITLE PAGE
%----------------------------------------------------------------------------------------

\begin{titlepage}
\begin{center}

\textsc{\Large \univname}\\
\textsc{\Large \facname}\\
\textsc{\Large \schoolname}\\[1cm]
%Path relative to the .tex file containing the \includegraphics command
\graphicspath{ {img/} }
\includegraphics[scale=.3]{logo.png} \\
\centering{
  \Large \bfseries
  Mejora en la optimización de la ubicación de trampas para mosquito con
  Particle Swarm Optimization.
}\\[0.5cm]

\large por\\[0.5cm]

\begin{minipage}{0.4\textwidth}
\begin{center} \large
\authors{Topiltzin Hernández Mares} %modificame  pon tu nombre
\large{\href{mailto:a01703266@tec.mx?subject=Awesome thesis, man!}{\authornames}} 
\\[0.5cm] 
\end{center}
\end{minipage}\\[0.5cm]

\large Proyecto Integrador para el Desarrollo de Soluciones Empresariales \\ \textit{\degreename}\\[0.8cm]

\begin{table}[!h]
\begin{center}
\begin{tabular}{lll}
\multicolumn{1}{r}{Asesor:} & Dr. Benjamín Valdés \\  %modificame nombre de tu asesor
\multicolumn{1}{r}{Co-Asesor:} & Dr. Héctor Sánchez \\
\end{tabular}
\end{center}
\end{table}

\vspace{3cm}
Santiago de Querétaro, Querétaro, México \\

{\large \longdate{08/06/2022}}\\[1.5cm]
\vfill
\end{center}

\end{titlepage}


\clearpage % Start a new page

%----------------------------------------------------------------------------------------
%	QUOTATION PAGE
%----------------------------------------------------------------------------------------

% \pagestyle{empty} % No headers or footers for the following pages%
% \null\vfill

% \begin{center}
% \textit{"the state is a plurality which should be united and made into a community by education"}
% \end{center}

% \begin{flushright}
% The Politics, Aristotle
% \end{flushright}


% \begin{center}
% \textit{"Nobody phrases it this way, but I think that artificial intelligence is almost a humanities discipline, ... It's really an attempt to understand human intelligence and human cognition."} 
% \end{center}
% \begin{flushright}
% Sebastian Thrun 
% \end{flushright}


% \vfill\vfill\vfill\vfill\vfill\vfill\null 

% \clearpage


%----------------------------------------------------------------------------------------
%	ACKNOWLEDGEMENTS
%----------------------------------------------------------------------------------------

\setstretch{1.3} % Reset the line-spacing to 1.3 for body text (if it has changed)

\acknowledgements{\addtocontents{toc}{\vspace{1em}}
  Aunque yo soy el autor de éste trabajo, no todo el esfuerzo fue mío, pues
  diversas personas contribuyeron e invirtieron su tiempo al desarrollo de este
  trabajo.
  
  Primeramente, quiero reconocer a mis padres, por apoyarme y siempre estar para
  mi cuando más los necesité. A mi hermana, por escucharme y hacerme ver las
  cosas desde otra perspectiva.

  Igualmente, quiero reconocer el esfuerzo de mis asesores, por resolver dudas y
  asistirme con diferentes temas durante la investigación. 

  Y finalmente, quiero agradecer a mis compañeros de clase (Alberto, Juan,
  Peter, Yaf, entre otros) por todas las pláticas y discusiones académicas, pues
  me ayudaron a llegar a mejores explicaciones que terminaron en este trabajo.
}
\clearpage % Start a new page


%----------------------------------------------------------------------------------------
%	ABSTRACT PAGE
%----------------------------------------------------------------------------------------

\addtotoc{Abstract} % Add the "Abstract" page entry to the Contents
\abstract{\addtocontents{toc}{\vspace{1em}}
A pesar de un incremento en las aplicaciones de algoritmos de optimización en
los últimos años, no existen muchas aplicaciones en el ámbito del control y
monitoreo de mosquitos. Actualmente solo existe MGSurvE, un paquete diseñado
para optimizar la posición de trampas para mosquito, sin embargo, su algoritmo
de optimización no es el más adecuado para esta tarea. Se desarrolló un
algoritmo PSO que busca mejorar los resultados arrojados por MGSurvE. Se
registraron mejoras en la cantidad de generaciones necesarias y en los
resultados mínimos de detección de mosquitos. 

\clearpage

%----------------------------------------------------------------------------------------
%	LIST OF CONTENTS/FIGURES/TABLES PAGES
%----------------------------------------------------------------------------------------

\pagestyle{fancy}

%\lhead{\emph{List of Figures}} % Set the left side page header to "List of Figures"
%\listoffigures % Write out the List of Figures

%\lhead{\emph{List of Tables}} % Set the left side page header to "List of Tables"
%\listoftables % Write out the List of Tables

\lhead{\emph{Contents}} % Set the left side page header to "Contents"
\tableofcontents % Write out the Table of Contents

%----------------------------------------------------------------------------------------
%	ABBREVIATIONS
%----------------------------------------------------------------------------------------

\clearpage
\setstretch{1.5}

\lhead{\emph{Acrónimos}} % Set the left side page header to "Abbreviations"
\listofsymbols{ll} % Include a list of Abbreviations (a table of two columns)
{
  \textbf{EA} & \textbf{E}volutionary \textbf{A}lgorithms \\
  \textbf{GA} & \textbf{G}enetic \textbf{A}lgorithm \\
  \textbf{PSO} & \textbf{P}article \textbf{S}warm \textbf{O}ptimization \\
  \textbf{SI} & \textbf{S}warm \textbf{I}ntelligence \\
  \textbf{MGSurvE} & \textbf{M}osquito \textbf{G}ene \textbf{Surv}eilance \textbf{E}xplorer \\
}

%----------------------------------------------------------------------------------------
%	THESIS CONTENT - CHAPTERS
%----------------------------------------------------------------------------------------

\mainmatter % Begin page numbering

\pagestyle{fancy}

% Include the chapters of the thesis as separate files from the Chapters folder

\input{capitulos}

%----------------------------------------------------------------------------------------
%	THESIS CONTENT - APPENDICES
%----------------------------------------------------------------------------------------

\addtocontents{toc}{\vspace{2em}}

\appendix % Cue to tell LaTeX that the following 'chapters' are Appendices

% Include the appendices of the thesis as separate files from the Appendices folder


% \input{apendices}  %modificame si not tienens apendices comenta esta linea

\addtocontents{toc}{\vspace{2em}}

\backmatter

%----------------------------------------------------------------------------------------
%	BIBLIOGRAPHY
%----------------------------------------------------------------------------------------
\label{Bibliography}

\lhead{\emph{Bibliografía}}
\bibliographystyle{unsrt}
\begin{thebibliography}{99}
  \bibitem{AIModernAproach}S. Russell and P. Norving, “Artificial Intelligence: A modern approach”, 3rd ed. Prentice Hall, New Jersey: Pearson Education, 2009. 
  \bibitem{AIDef}J. M. Spector and S. Ma, “Inquiry and critical thinking skills for the next generation: From Artificial Intelligence back to human intelligence,” Smart Learning Environments, vol. 6, no. 1, Sep. 2019. 
  \bibitem{SearchMethodologies}E. K. Burke and G. Kendall, “Search methodologies: Introductory tutorials in optimization and decision support techniques”. New York, New York: Springer, 2014.
  \bibitem{HeuristicDef}V. J. Rayward-Smith, “Modern Heuristic Search Methods”. Chichester, New York: Wiley, 1996. 
  \bibitem{MetaheuristicsDef}F. Glover and M. Laguna, “Tabu Search”. Boston, Massachusets: Kluwer Academic Publishers, 1997.
  \bibitem{EADef}C. A. Coello Coello, G. B. Lamont, and D. A. Van Veldhuizen, “Evolutionary algorithms for solving multi-objective problems,” Genetic and Evolutionary Computation Series, 2007. 
  \bibitem{GADef}A. S. Fraser, “Simulation of genetic systems by Automatic Digital Computers II. effects of linkage on rates of advance under selection,” Australian Journal of Biological Sciences, vol. 10, no. 4, p. 492, 1957. 
  \bibitem{PSODef}J. Kennedy and R. Eberhart, “Particle swarm optimization,” Proceedings of ICNN'95 - International Conference on Neural Networks, 1995. 
  \bibitem{CPSO}Y. Shi and R. Eberhart, “A modified particle swarm optimizer,” 1998 IEEE International Conference on Evolutionary Computation Proceedings. IEEE World Congress on Computational Intelligence (Cat. No.98TH8360), 1998. 
  \bibitem{PSOReview}S. Cheg, H. Lu, X. Lei, and Y. Shi, “A quarter century of particle swarm optimization,”, Complex and Intelligent Systems, vol. 4, no. 3, pp. 227 239, 2018.

  \bibitem{3DAFAO}S. Li, C. Zhang, and J. Qu, “Location optimization of wireless sensor network in Intelligent Workshop based on the three-dimensional adaptive fruit fly optimization algorithm,” International Journal of Online Engineering (iJOE), vol. 14, no. 11, p. 202, Nov. 2018. 
  \bibitem{PSOEnergy}N. Hantash, T. Khatib, and M. Khammash, “An improved particle swarm optimization algorithm for optimal allocation of distributed generation units in Radial Power Systems,” Applied Computational Intelligence and Soft Computing, vol. 2020, pp. 1-8.
  \bibitem{APSO2016}H. Liang and F. H. Kang, “Adaptive mutation particle swarm algorithm with dynamic nonlinear changed inertia weight,” Optik-International Journal for Light and Electron Optics, vol. 127, no. 19, pp. 8036-8042, 2016.
  \bibitem{MGSurvE}H. M. Sanchez, “MGSurvE's documentation,” MGSurvE's documentation! - MGSurvE documentation, 2021. [Online]. Available: https://chipdelmal.github.io/MGSurvE/build/html/index.html. [Accessed: 31-Mar-2022]. 
  \bibitem{DEAP}DEAP Project, “DEAP documentation,” DEAP documentation - DEAP 1.3.1 documentation, 2022. [Online]. Available: https://deap.readthedocs.io/en/master/. [Accessed: 01-Apr-2022]. 
  \bibitem{DE&PSOCov}A. P. Piotrowski and A. E. Piotrowska, “Differential Evolution and particle swarm optimization against covid-19,” Artificial Intelligence Review, Jul. 2021. 
  \bibitem{SwarmVsEvol}A. P. Piotrowski, M. J. Napiorkowski, J. J. Napiorkowski, and P. M. Rowinski, “Swarm intelligence and Evolutionary Algorithms: Performance versus speed,” Information Sciences, vol. 384, pp. 34-85, Apr. 2017. 
  \bibitem{PSOPopulationSize}A. P. Piotrowski, J. J. Napiorkowski, and A. E. Piotrowska, “Population size in particle swarm optimization,” Swarm and Evolutionary Computation, vol. 58, May 2020. 
  \bibitem{MarshallLab}The Marshall Lab at UC Berkeley. (s. f.). THE MARSHALL LAB. https://www.marshalllab.com/

  \bibitem{ShapeUp}R. Singer, “Shape up V 1.8, 2019 edition,” Shape Up: Stop Running in Circles and Ship Work that Matters, 2019. [Online]. Available: https://basecamp.com/shapeup/webbook. [Accessed: 03-May-2022]. 
  \bibitem{PandasDocs}“Pandas,” pandas, 02-Apr-2022. [Online]. Available: https://pandas.pydata.org/. [Accessed: 06-May-2022]. 
  \bibitem{NumPyDocs}NumPy, 2022. [Online]. Available: https://numpy.org/. [Accessed: 06-May-2022]. 
  \bibitem{DEAPDocs}“DEAP documentation,” DEAP documentation - DEAP 1.3.1 documentation, 22-Jan-2022. [Online]. Available: https://deap.readthedocs.io/en/master/. [Accessed: 06-May-2022]. 
\end{thebibliography}
\end{document}
